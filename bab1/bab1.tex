
\chapter{PENDAHULUAN}
\label{cha:1-Pendahuluan}

\section{Latar Belakang}
\label{sec:1-LatarBelakang}

Pemanfaatan teknologi digital pada era modern selalu meninggalkan jejak digital dalam bentuk data 
digital. Bentuk data digital yang paling berguna bagi manusia adalah teks, citra audio, dan citra 
visual. Jejak digital ini akan selalu menjadi acuan dalam langkah pengambilan keputusan dalam 
berbagai bidang baik dari lingkup nasional maupun internasional. Penggunaan data digital yang ada 
juga harus mengikuti peraturan perundang-undangan yang berlaku sehingga produk dan perangkat lunak 
yang dihasilkan jauh dari tindak penyalahgunaan.

Pemelajaran mesin atau \textit{machine learning} merupakan mekanisme yang memungkinkan komputer 
untuk mengenali pola kompleks dari data empiris secara otomatis. Data empiris memiliki keterkaitan 
antara variabel independen dan variabel dependen. Penelitian ini mengacu kepada pose dua dimensi 
sebagai variabel independen dan pose tiga dimensi sebagai variabel dependen. Pemelajaran mesin 
dapat melakukan pemetaan antara kedua variabel ini dalam bentuk jaringan saraf tiruan dalam atau 
\textit{deep neural network}.

\textit{Residual Network (ResNet)} merupakan salah satu arsitektur yang termasuk kedalam golongan 
\textit{deep neural network}. Arsitektur ini menerapkan skema \textit{skip connection} yang sejauh 
ini metode terbaik dalam melakukan pemelajaran mesin. Pendekatan menggunakan \textit{ResNet} 
memungkinkan pemetaan pose dua dimensi ke pose tiga dimensi pada setiap titik kunci spesifik dengan 
tingkat akurasi yang relatif tinggi.

\section{Batasan Masalah}
\label{sec:1-BatasMasalah}

Penelitian ini menganggap setiap pose dua dimensi maupun pose tiga dimensi berada dalam koordinat 
lokal. Setiap pose ditransformasi ke dalam observasi kamera dengan titik kunci pinggang 
sebagai posisi tengah. Hal ini dilakukan karena pemetaan hanya menggunakan grafik datar tanpa 
informasi kedalaman titik kunci. Hal ini dilakukan untuk menghindari masalah kedalaman yang ambigu. 

\section{Tujuan Penelitian}
\label{sec:1-TujuanPenelitian}

Tujuan dari penelitian ini adalah untuk membuat sebuah aplikasi yang dapat membaca titik kunci dari 
sebuah citra visual datar dan mentransformasikannya kedalam pose lokal dua dimensi sehingga dapat 
dipetakan oleh jaringan saraf tiruan yang dimodelkan ke bentuk pose lokal tiga dimensi. Pose hasil
juga divisualisasikan secara interaktif sehingga dapat dipergunakan untuk kepentingan yang sesuai.

\section{Metode Penelitian}
\label{sec:1-MetodePenelitian}

Penelitian dibagi menjadi tiga tahap besar yang terdiri dari \textit{data preprocessing}, 

\section{Sistematika Penulisan}
\label{sec:1-SistematikaPenulisan}
