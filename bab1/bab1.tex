
\chapter{PENDAHULUAN}
\label{cha:1-Pendahuluan}

\section{Latar Belakang}
\label{sec:1-LatarBelakang}

Pemanfaatan teknologi digital pada era modern selalu meninggalkan jejak digital dalam bentuk data 
digital. Bentuk data digital yang paling berguna bagi manusia adalah teks, citra audio, dan citra 
visual. Jejak digital ini akan selalu menjadi acuan dalam langkah pengambilan keputusan dalam 
berbagai bidang baik dari lingkup nasional maupun internasional. Penggunaan data digital yang ada 
juga harus mengikuti peraturan perundang-undangan yang berlaku sehingga produk dan perangkat lunak 
yang dihasilkan jauh dari tindak penyalahgunaan.

Pemelajaran mesin atau \textit{machine learning} merupakan mekanisme yang memungkinkan komputer 
untuk mengenali pola kompleks dari data empiris secara otomatis. Data empiris memiliki keterkaitan 
antara variabel independen dan variabel dependen. Penelitian ini mengacu kepada pose dua dimensi 
sebagai variabel independen dan pose tiga dimensi sebagai variabel dependen. Pemelajaran mesin 
dapat melakukan pemetaan antara kedua variabel ini dalam bentuk jaringan saraf tiruan.



\section{Batasan Masalah}
\label{sec:1-BatasMasalah}

Bagian ini menceritakan tentang :
\begin{itemize}
\item Batasan penelitian beserta alasannya.
\item Definisi permasalahan dari penelitian
\item Tujuan umum dan khusus dari penelitian
\end{itemize}

\section{Tujuan Penelitian}
\label{sec:1-TujuanPenelitian}

\section{Metode Penelitian}
\label{sec:1-MetodePenelitian}

\section{Sistematika Penulisan}
\label{sec:1-SistematikaPenulisan}
