
\chapter{PENDAHULUAN}
\label{cha:1-Pendahuluan}

\section{Latar Belakang Masalah}
\label{sec:1-LatarBelakangMasalah}

Pemanfaatan teknologi yang terkomputerisasi oleh manusia selalu meninggalkan jejak yang
tersimpan dalam bentuk data digital. Rekam jejak ini merupakan bukti perilaku dan karakteristik
manusia sehingga dijadikan sebagai acuan pengembangan teknologi dan ilmu pengetahuan pada masa mendatang.
Data digital yang umumnya dimanfaatkan oleh manusia meliputi teks, citra audio, citra visual, dan
citra audio visual yang disimpan ke dalam suatu media penyimpanan. Banyaknya jumlah data yang tersedia
dan diprediksi akan semakin bertambah membuat gaya hidup manusia semakin
bergantung pada teknologi digital.

Jejak digital bersifat laten yang berarti data digital memiliki makna khusus dan hanya dapat diolah
dengan prosedur khusus. Citra visual terbentuk dari penangkapan pantulan gelombang elektromagnetik
benda yang berada depan kamera dan kemudian disimpan dalam bentuk digital. Hasil akhir citra visual
yang tercipta hanya memiliki informasi numerik mengenai warna, sedangkan
informasi posisi benda saat penangkapan sudah hilang. Estimasi informasi laten dapat
dilakukan komputer dengan membuat fungsi pemetaan dari suatu gambar terhadap suatu kondisi dengan
memanfaatkan data dalam jumlah besar sebagai acuan.

Permodelan pemelajaran dalam atau \textit{deep learning} dapat memetakan suatu domain ke
domain lainnya secara mandiri menggunakan pemelajaran jaringan saraf tiruan dalam atau
\textit{deep neural network}. Pemelajaran dalam dapat dilakukan dengan komputasi mandiri yang
sangat bergantung pada kuantitas dan kualitas data yang baik.
Pemelajaran dalam menggunakan jaringan
saraf tiruan dapat digunakan untuk mengembangkan teknologi khususnya di bidang visi komputer
seperti melakukan estimasi pose tiga dimensi tubuh manusia yang terdapat dalam suatu gambar monokuler.

\section{Rumusan Masalah}
\label{sec:1-RumusanMasalah}
Permasalahan yang ingin diselesaikan adalah mengestimasi pose tubuh manusia yang
bersifat laten dengan membuat sebuah fungsi pemetaan dari gambar yang ditangkap kamera monokuler ke
koordinat setiap titik kunci pose menggunakan model \textit{deep neural network}.
Model yang telah terlatih mampu melakukan proses
inferensi terhadap gambar baru sehingga dapat dijadikan sebagai acuan aplikasi yang mampu membaca pose tubuh
manusia dari gambar monokuler.

\section{Batasan Masalah}
\label{sec:1-BatasMasalah}

Penelitian ini menganggap setiap pose dua dimensi maupun pose tiga dimensi berada dalam koordinat
lokal.
Pose yang dihasilkan tidak bersifat \textit{grounded} yang berarti pose tidak memperhitungkan lantai
sebagai titik tengah.
Setiap pose ditransformasi ke dalam sistem koordinat kamera dengan titik kunci pinggang
sebagai titik koordinat tengah. Hal ini dilakukan karena pemetaan hanya menggunakan gambar dua dimensi
tanpa informasi kedalaman titik kunci sehingga dapat mengesampingkan masalah kedalaman yang ambigu.

\section{Tujuan Penelitian}
\label{sec:1-TujuanPenelitian}

Tujuan dari penelitian ini adalah untuk membuat sebuah aplikasi yang dapat mengestimasi titik kunci dari
sebuah citra visual datar dan melakukan transformasi ke pose lokal dua dimensi sehingga dapat
dipetakan oleh jaringan saraf tiruan yang dimodelkan ke bentuk pose lokal tiga dimensi. Pose hasil
juga divisualisasikan secara interaktif sehingga dapat digunakan untuk kepentingan yang sesuai.

\section{Metode Penelitian}
\label{sec:1-MetodePenelitian}

Penelitian dibagi menjadi tiga tahap besar terurut yang terdiri dari \textit{data preprocessing},
pelatihan model jaringan saraf tiruan, dan visualisasi. Tahap pertama dan tahap ketiga tidak
melibatkan pemelajaran mesin. Permasalahan utama dari penelitian ini berada pada tahap kedua tentang
pelatihan jaringan saraf tiruan untuk melakukan pemetaan pose dua dimensi ke pose tiga dimensi.

\textit{Dataset} yang digunakan dalam penelitian ini adalah \textit{Human3.6M} yang berisi 3,6 juta
pose unik yang dilakukan oleh sebelas peraga profesional dan direkam menggunakan empat sudut kamera
yang berbeda beserta dengan koordinat setiap titik kunci dari hasil penangkapan alat
\textit{motion capture}~\cite{h36m_pami}.

Pose dua dimensi dan tiga dimensi merupakan variabel yang relevan untuk masalah. Pada tahap
\textit{data preprocessing} akan dilakukan ekstraksi variabel ini menjadi bentuk numerik sehingga
mudah untuk digunakan saat melakukan pelatihan jaringan saraf tiruan. Tahap selanjutnya akan
dilakukan pembuatan, permodelan, pelatihan, dan evaluasi jaringan saraf tiruan dalam untuk
pemetaan titik kunci yang kemudian dilanjutkan dengan percobaan model dengan video. Tahap terakhir
menampilkan visualisasi gambar, pose dua dimensi, dan pose tiga dimensi.

Perangkat keras yang digunakan dalam penelitian ini adalah satu unit laptop dengan spesifikasi:
\begin{itemize}
    \item CPU Intel Core I7 7700HQ
    \item Memori 24 GB DDR4
    \item GPU NVIDIA GTX 1060 6GB
    \item SSD NVME SAMSUNG 120 GB
    \item HDD SATA 1 TB
\end{itemize}

Perangkat lunak yang digunakan dalam penelitian ini adalah satu unit laptop dengan spesifikasi:
\begin{itemize}
    \item Python 3.7
    \item Jupyter Lab
    \item Git
    \item GitHub
    \item LaTeX
\end{itemize}

\section{Sistematika Penulisan}
\label{sec:1-SistematikaPenulisan}

Adapun sistematika penulisan yang digunakan dalam penulisan ini adalah sebagai berikut:

PENDAHULUAN, mengemukakan latar belakang masalah, rumusan masalah, batasan masalah, tujuan
penelitian, metode penelitian, dan sistematika penulisan.

TINJAUAN PUSTAKA, menjelaskan kumpulan teori yang digunakan dalam mendukung proses penyelesaian
program.

PENDEKATAN, mengemukakan langkah-langkah yang dicapai untuk membuat program.

HASIL DAN ANALISIS, mendalami hasil yang tercapai dengan analisis secara mendalam.

PENUTUP, mengulas lebih lanjut mengenai kesimpulan yang dapat ditarik dari hasil disertai dengan
saran yang dapat menyempurnakan penelitian selanjutnya.