
\chapter{PENDAHULUAN}
\label{cha:1-Pendahuluan}

\section{Latar Belakang}
\label{sec:1-LatarBelakang}

Pemanfaatan teknologi yang terkomputerisasi oleh manusia selalu meninggalkan rekam jejak yang
tersimpan dalam bentuk data digital. Data digital yang umumnya dimanfaatkan oleh manusia
meliputi teks, citra audio, citra visual, dan citra audio visual yang dimampatkan kedalam suatu
media penyimpanan. Pengaksesan data ini tergolong mudah karena data dikonstruksi dengan struktur
dan format yang sama sehingga dapat diakses dimana saja. Jumlah rekam jejak digital yang tersedia
semakin bertambah seiring dengan mudahnya akses terhadap teknologi secara skala besar.

Rekam jejak digital yang terkumpul bersifat laten yang berarti data yang tersedia tidak
memiliki semantik. Data ini hanya akan berguna apabila ada suatu aturan yang ditentukan
oleh pengguna. Proses pembuatan citra visual terbentuk dari hasil perekaman kamera dengan mengubah
gelombang elektromagnetik yang dipantulkan oleh benda dari posisi tertentu menuju lensa kamera
menjadi tiga lapis deretan angka numerik yang mewakili warna merah, hijau, dan biru. Citra visual
yang tercipta hanya memiliki informasi numerik mengenai warna dalam bentuk gambar, sedangkan
informasi posisi benda saat perekaman sudah hilang. Otak manusia dapat mengartikan kondisi suatu
benda dalam suatu gambar tanpa kita ketahui cara kerjanya secara pasti. Hal yang sama dapat
dilakukan oleh komputer dengan membuat pemetaan dari suatu gambar ke suatu kondisi yang diinginkan.

Permasalahan yang ingin diselesaikan adalah mengembalikan informasi posisi pose tubuh manusia yang
telah hilang dengan membuat pemetaan dari gambar yang ditangkap menggunakan kamera monokuler ke
koordinat setiap titik kunci dari pose tersebut.
Permodelan pemelajaran dalam atau \textit{deep learning} dapat pemetaan suatu domain ke
domain lainnya secara otomatis menggunakan pelatihan jaringan saraf tiruan dalam atau
\textit{deep neural network}. Fungsi pemetaan dapat dihasilkan dengan melatih jaringan saraf tiruan
dengan jumlah data yang besar.

\section{Batasan Masalah}
\label{sec:1-BatasMasalah}

Penelitian ini menganggap setiap pose dua dimensi maupun pose tiga dimensi berada dalam koordinat
lokal. Setiap pose ditransformasi ke dalam observasi kamera dengan titik kunci pinggang
sebagai posisi tengah. Hal ini dilakukan karena pemetaan hanya menggunakan grafik datar tanpa
informasi kedalaman titik kunci sehingga dapat menghindari masalah kedalaman yang ambigu.

\section{Tujuan Penelitian}
\label{sec:1-TujuanPenelitian}

Tujuan dari penelitian ini adalah untuk membuat sebuah aplikasi yang dapat membaca titik kunci dari
sebuah citra visual datar dan melakukan transformasi ke pose lokal dua dimensi sehingga dapat
dipetakan oleh jaringan saraf tiruan yang dimodelkan ke bentuk pose lokal tiga dimensi. Pose hasil
juga divisualisasikan secara interaktif sehingga dapat dipergunakan untuk kepentingan yang sesuai.

\section{Metode Penelitian}
\label{sec:1-MetodePenelitian}

Penelitian dibagi menjadi tiga tahap besar terurut yang terdiri dari \textit{data preprocessing},
pelatihan model jaringan saraf tiruan, dan visualisasi. Tahap pertama dan tahap ketiga tidak
melibatkan pemelajaran mesin. Masalah utama dari penelitian ini berada pada tahap kedua tentang
pelatihan jaringan saraf tiruan untuk melakukan pemetaan pose dua dimensi ke pose tiga dimensi.

\textit{Dataset} yang digunakan dalam penelitian ini adalah \textit{Human3.6M} yang berisi 3,6 juta
pose unik yang dilakukan oleh sebelas aktor profesional dan direkam menggunakan empat sudut kamera
yang berbeda beserta dengan koordinat setiap titik kunci dari hasil penangkapan alat
\textit{motion capture}~\cite{h36m_pami}.

Pose dua dimensi dan tiga dimensi merupakan variabel yang relevan untuk masalah. Pada tahap
\textit{data preprocessing} akan dilakukan ekstraksi variabel ini kedalam bentuk numerik sehingga
mudah untuk digunakan saat melakukan pelatihan jaringan saraf tiruan. Tahap selanjutnya akan
dilakukan pembuatan, permodelan, pelatihan, dan evaluasi jaringan saraf tiruan dalam untuk
pemetaan titik kunci yang kemudian dilanjutkan dengan percobaan model dengan video. Tahap terakhir
menampilkan visualisasi gambar, pose dua dimensi, dan pose tiga dimensi.

Perangkat keras yang digunakan dalam penelitian ini adalah satu unit laptop dengan spesifikasi:
\begin{itemize}
  \item CPU Intel Core I7 7700HQ
  \item Memori 24 GB DDR4
  \item GPU NVIDIA GTX 1060 6GB
  \item SSD NVME SAMSUNG 120 GB
  \item HDD SATA 1 TB
\end{itemize}

\section{Sistematika Penulisan}
\label{sec:1-SistematikaPenulisan}

Adapun sistematika penulisan yang digunakan dalam penulisan ini adalah sebagai berikut:

PENDAHULUAN, mengemukakan latar belakang masalah, batasan masalah, tujuan penelitian, metode
penelitian, dan sistematika penulisan.

TINJAUAN PUSTAKA, menjelaskan kumpulan teori yang digunakan dalam mendukung proses penyelesaian
program.

PENDEKATAN, mengemukakan langkah-langkah yang dicapai untuk membuat program.

HASIL DAN ANALISIS, menghubungkan hasil yang didapatkan dengan teori-teori yang dibahas pada bab
\ref{cha:2-TinjauanPustaka}.

PENUTUP, mengulas lebih lanjut mengenai kesimpulan yang dapat ditarik dari hasil disertai dengan
saran yang dapat menyempurnakan penelitian selanjutnya.