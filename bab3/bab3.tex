
\chapter{PENDEKATAN}
\label{cha:3-Pendekatan}

\section{Gambaran Umum} \label{sec:3-GambaranUmum}

Penelitian ini membahas pemanfaatan data gambar sebagai acuan dalam melakukan pelatihan dan
implementasi model \textit{deep neural network} untuk mencari dan memetakan koordinat tiga dimensi
pose tubuh manusia dalam sebuah rangkaian gambar secara lokal. Pengerjaan aplikasi mengutamakan dua
langkah penting yang meliputi pengolahan data dan pembuatan model. Aplikasi yang dibuat dapat menampilkan
plot grafik tiga dimensi menyerupai struktur anatomi tubuh manusia sesuai dengan pose hasil
estimasi dari gambar masukkan. Hasil pelatihan model ditampilkan dalam grafik dua dimensi untuk
analisis lebih lanjut.

Data yang digunakan dalam penelitian ini terbagi menjadi tiga jenis meliputi data pelatihan model,
data uji coba model, dan data implementasi aplikasi.

implementasi aplikasi dari gambar 2d ke plot.

\section{Kerangka Penelitian} \label{sec:3-KerangkaPenelitian}

\section{Perancangan Aplikasi} \label{sec:3-PerancanganAplikasi}

\subsection{Skema Pelatihan Model}

\subsection{Skema Penerapan Aplikasi}

\section{Analisis Data} \label{sec:3-AnalisisData}

\subsection{Pra-Pemrosesan Data}

\subsection{Definisi Titik Kunci}
COCO Keypoints, Custom Keypoints

\section{Pembuatan Model} \label{sec:3-PerancanganModel}
\subsection{Model A}
\subsection{Model B}
\subsection{Model C}

\section{Pelatihan Model} \label{sec:3-PelatihanModel}

\begin{table}[htbp]
    \captionsetup{labelfont=bf, textfont=bf}
    \caption{Sebuah tabel}
    \vspace{-20pt}
    \begin{center}
        \begin{tabular}{| l c r |}
            \hline
            1 & 2 & 3 \\
            4 & 5 & 6 \\
            7 & 8 & 9 \\
            \hline
        \end{tabular}
    \end{center}
    \vspace{-10pt}
    \captionsetup{labelfont=md, textfont=md}
    % \caption*{Sumber: Bego Lu}
\end{table}