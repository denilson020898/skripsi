

\chapter{TINJAUAN PUSTAKA}
\label{cha:2-TinjauanPustaka}

\section{Teorema Penaksiran Universal \label{sec:2-TeoremaPenaksiranUniversal}}
Berisi tinjauan Teorema Penaksiran Universal~\cite{fastbook} \cite{8765346}

\section{Pemelajaran Mesin\label{sec:2-PemelajaranMesin}}
Apa itu pemelajaran mesin / machine learning.
\begin{enumerate}
  \item Klasifikasi (probabilitas diskrit)
  \begin{itemize}
    \item Binary Label
    \item Multi Label
    \item Multi Class
  \end{itemize}
  \item Regresi (Kuantitas Numerik)
\end{enumerate}

\section{Jaringan Sarah Tiruan \label{sec:2-JaringanSarafTiruan}}
Berisi tinjauan Jaringan Saraf Tiruan

\begin{itemize}
  \item model
  \item arsitektur
  \item input
  \item parameter
  \item variabel independen
  \item label
  \item variable dependen
  \item hasil/prediksi
  \item loss
  \item optimizer
\end{itemize}

Batasan:
\begin{itemize}
  \item Model tidak bisa dibuat tanpa data
  \item Model hanya bisa mempelajari pola dari data yang digunakan.
  \item Hanya menghasilkan prediksi/ramalan, bukan rekomendasi keputusan
  \item Memerlukan data input dan label yang umumnya dibuat manual
\end{itemize}


\section{Transfer Learning\label{sec:2-TransferLearning}}
Berisi tinjauan Transfer Learning

\section{Tinjauan N \label{sec:2-TinjauanN}}
Berisi tinjauan N

\section{Perbandingan Tinjauan \label{sec2-Banding}}
Membandingkan dengan melihat kelebihan kekurangan dari masing-masing tinjauan, dan pilihan mana 
yang digunadakan atau diadaptasi.
