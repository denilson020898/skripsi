\newpage %Acknowledgment
\addcontentsline{toc}{chapter}{KATA PENGANTAR}
\begin{center}
  \begin{large}\textbf{KATA PENGANTAR}\\\end{large}
\end{center}
\vspace{5mm}


Segala puji dan syukur penulis ucapkan ke hadirat Tuhan Yang Maha Esa yang telah memberikan berkat,
anugerah dan karunia yang melimpah, sehingga penulis dapat menyelesaikan tugas akhir ini pada waktu
yang telah ditentukan.

Tugas akhir ini disusun guna melengkapi sebagian syarat untuk memperoleh gelar Sarjana Teknik
Informatika Universitas Gunadarma. Adapun judul tugas akhir ini adalah "Estimasi Pose Tiga Dimensi
Dari Gambar Monokuler Menggunakan Deep Neural Network".

Walaupun banyak kesulitan yang penulis harus hadapi ketika menyusun tugas akhir ini, namun berkat
bantuan dan dorongan dari berbagai pihak, akhirnya tugas akhir ini dapat diselesaikan dengan baik.
Untuk itu penulis tidak lupa mengucapkan terima kasih kepada:

\begin{enumerate}
  \item Prof. Dr. E. S. Margianti, SE., MM., selaku Rektor Universitas Gunadarma.
  \item Prof. Dr. -Ing. Adang Suhendra, SSi., S.Kom., M.Sc., selaku Dekan Fakultas Teknologi Industri Universitas Gunadarma.
  \item Dr. Lintang Yuniar Banowosari, S.Kom., M.Sc., selaku Ketua Jurusan Fakultas Teknik Informatika Universitas Gunadarma.
  \item Dr, Edi Sukirman, SSi., MM., selaku Kepala Bagian Sidang Ujian Universitas Gunadarma.
  \item Dr. Dharmayanti, ST., MMSI., sebagai dosen pembimbing penulis yang ditengah-tengah kesibukannya
        telah membimbing penulis sehingga penulisan ini dapat diselesaikan.
  \item Keluarga yang selalu mendukung dan terus memberikan motivasi.
  \item Semua pihak yang terlibat dalam membantu penyelesaian Tugas Akhir ini.

\end{enumerate}

Sebagai manusia biasa yang tak luput dari kesalahan, maka penulis meminta maaf atas segala
kekurangan dan keterbatasan dalam penyusunan tugas akhir ini. Penulis sadari bahwa penulisan ini
masih jauh dari sempurna, disebabkan karena berbagai keterbatasan yang penulis miliki. Untuk itu
penulis mengharapkan kritik dan saran yang bersifat membangun untuk menjadi perbaikan di masa yang
akan datang.

Akhir kata, penulis berharap penulisan ini dapat bermanfaat bagi semua pihak dan bagi penulis
pribadi khususnya, serta dapat digunakan sebagaimana mestinya.


\vspace{0.5 cm}
\begin{flushright}
  Jakarta, April 2020

  \vspace{2 cm}
  Denilson
\end{flushright}