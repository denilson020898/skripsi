\newpage %Acknowledgment
\addcontentsline{toc}{chapter}{KATA PENGANTAR}
\begin{center}
\begin{large}\textbf{KATA PENGANTAR}\\\end{large}
\end{center}
\vspace{5mm}


Segala puji dan syukur penulis naikkan ke hadirat Tuhan Yang Maha Esa yang telah memberikan berkat, 
anugerah dan karunia yang melimpah, sehingga penulis dapat menyelesaikan Tugas Akhir ini pada waktu 
yang telah ditentukan.

Tugas Akhir ini disusun guna melengkapi sebagian syarat untuk memperoleh gelar Sarjana Teknik 
Informatika Universitas Gunadarma. Adapun judul Tugas Akhir ini adalah "Estimasi Pose Tiga Dimensi
Dari Gambar Monokuler Menggunakan Deep Neural Network".

Walaupun banyak kesulitan yang penulis harus hadapi ketika menyusun Tugas Akhir ini, namun berkat 
bantuan dan dorongan dari berbagai pihak, akhirnya Tugas Akhir ini dapat diselesaikan dengan baik. 
Untuk itu penulis tidak lupa mengucapkan terima kasih kepada:

\begin{enumerate}
  \item Ibu Prof. E. S. Margianti, SE, MM selaku rektor Universitas Gunadarma
  \item ............ selaku Dekan Fakultas ........ Universitas Gunadarma
  \item ............ selaku Ketua Jurusan ..............
  \item ............ selaku Bagian Sidang Sarjana
  \item Ibu Dr. Dharmayanti, ST., MMSI sebagai pembimbing penulis yang ditengah-tengah kesibukannya 
  telah membimbing penulis sehingga penulisan ini dapat diselesaikan.
  \item Keluarga yang selalu mendukung dan terus memberikan motivasi.
  \item Semua pihak yang terlibat dalam membantu penyelesaian Tugas Akhir ini.

\end{enumerate}

Sebagai manusia biasa yang tak luput dari kesalahan, maka penulis meminta maaf atas segala 
kekurangan dan keterbatasan dalam penyusunan Tugas Akhir ini. Penulis sadari bahwa penulisan ini 
masih jauh dari sempurna, disebabkan karena berbagai keterbatasan yang penulis miliki. Untuk itu 
penulis mengharapkan kritik dan saran yang bersifat membangun untuk menjadi perbaikan di masa yang 
akan datang.

Akhir kata, penulis berharap penulisan ini dapat bermanfaat bagi kita semua dan bagi penulis 
pribadi khususnya, serta dapat digunakan sebagaimana mestinya.


\vspace{0.5 cm}
\begin{flushright}
Depok, April 2020

\vspace{2 cm}
Penulis
\end{flushright}