\newpage %Abstract
\addcontentsline{toc}{chapter}{ABSTRAKSI}
\begin{center}
    \begin{large}\textbf{ABSTRAKSI}\end{large}
\end{center}

\vspace{5mm}

\noindent Denilson, 51416815 \\
ESTIMASI POSE TIGA DIMENSI DARI GAMBAR MONOKULER MENGGUNAKAN DEEP NEURAL NETWORK\\
Tugas Akhir. Jurusan Teknik Informatika, Fakultas Teknologi Industri, \\
Universitas Gunadarma, 2020\\
Kata Kunci: Estimasi Pose, Gambar Monokuler, Jaringan Saraf Tiruan, Pemelajaran Dalam, Visi Komputer\\
\noindent (xiv + 40 + lampiran)\\

\setstretch{1.0}
Perkembangan teknologi digital yang pesat baik pada aplikasi atau ilmu pengetahuan dapat
manghasilkan rekam jejak digital yang bermanfaat. Jumlah data digital yang tersedia sangat
banyak dan diprediksi akan semakin bertambah. Salah satu penggunaan data adalah membuat
suatu fungsi pemetaan yang mencari korelasi antara suatu domain ke domain lainnya dengan
menggunakan data terkait sebagai acuan dasar. Data digital berbentuk rangkaian gambar atau video
merupakan data yang bersifat laten yang berarti data tersebut memiliki informasi semantik yang
tersembunyi. Penelitian ini membahas pembuatan sebuah fungsi yang memetakan gambar dua dimensi
terhadap titik kunci pose tiga dimensi yang bersifat laten menggunakan permodelan
\textit{deep neural network}. Perangkat lunak yang dibangun dengan pemrosesan data,
perancangan arsitektur model, pemelajaran model secara mandiri, dan menampilkan visualisasi
penggunaan model. Arsitektur model yang digunakan terdiri dari beberapa blok \textit{residual network}
yang menambahkan \textit{input} terhadap \textit{output} masing-masing blok. Hasil dari uji coba menjelaskan
bahwa teori dan data yang dipakai benar dan penggunaan aplikasi terhadap data baru berjalan
sesuai prediksi.\\


\setstretch{1.5}
\noindent Daftar Pustaka (1986-2020)
