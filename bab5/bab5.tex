
\chapter{PENUTUP}
\label{cha:5-PENUTUP}

\section{Kesimpulan}
\label{sec:5-Kesimpulan}

Aplikasi estimasi pose tiga dimensi menggunakan model \textit{deep neural network} berhasil dilatih.
Model ini melakukan pemelajaran secara mandiri menggunakan informasi pose dua dimensi sebagai \textit{input}
dan pose tiga dimensi sebagai \textit{output}.
Model tersebut diuji untuk menemukan pose tiga dimensi pada data inferensi. Model melakukan pemelajaran
selama sepuluh \textit{epochs} dengan hasil rata-rata kesalahan akhir bernilai 0.0584 pada data pelatihan
dan 0.0437 pada data validasi. Nilai kesalahan data pelatihan masih lebih besar daripada nilai data validasi.
Hal ini menandakan model masih berada pada kondisi \textit{underfitting} dimana selisih kedua nilai tersebut relatif
besar. Model yang lebih baik dapat didapatkan dengan melatih model dalam jumlah \textit{epochs} yang
lebih banyak dan berhenti saat mulai terjadi \textit{overfitting}.

\section{Saran}
\label{sec:5-Saran}

Pengembangan model \textit{deep neural network} ini masih menggunakan arsitektur minimalis, data dengan
satu domain, dan memiliki tahapan yang tidak efisien. Pengembangan selanjutnya disarankan menggunakan
arsitektur yang lebih efisien. Arsitekur \textit{residual network} merupakan arsitektur yang paling
bagus pada saat penulisan ini dilakukan. Algoritma pemelajaran yang lebih efisien juga disarankan
pada penelitian mendatang. Data dengan domain yang lebih luas juga merupakan hal yang penting seperti
estimasi pose pada hewan tertentu. Dengan demikian penelitian ini dapat bermanfaat dan dapat dikembangkan
menjadi jauh lebih baik lagi pada masa mendatang.